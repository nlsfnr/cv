\documentclass[11pt]{article}

\usepackage{xcolor}
\usepackage{tabularx}

% Page layout
\usepackage{geometry}
\newcommand\MARGIN{20mm}
\geometry{
    a4paper,
    left=\MARGIN,
    right=\MARGIN,
    top=\MARGIN,
    bottom=\MARGIN,
}

% Do not indent the first line of paragraphs
\setlength{\parindent}{0cm}

% Layout stuff
\usepackage{lastpage}
\usepackage{fancyhdr}
\pagestyle{fancy}
\fancyhf{}
\renewcommand{\headrulewidth}{0pt}
\rfoot{Page \thepage \hspace{1pt} of \pageref{LastPage}}

% Custom section titles
\usepackage{titlesec}
\newcommand\TITLE[1]{\par{\centering{\Huge\textsc{#1}}\bigskip\par}}
\titleformat{\section}{\Large\scshape\raggedright}{}{0em}{}[\titlerule]
\titlespacing{\section}{0pt}{0.5cm}{0.3cm}

% URLs
\usepackage{hyperref}
\definecolor{linkcolour}{rgb}{0,0.15,0.4}
\hypersetup{colorlinks,breaklinks,urlcolor=linkcolour, linkcolor=linkcolour}

% Bulletpoints instead of itemize, as they do not work well inside tables
\newcommand{\tabitem}{{\textbullet}~}

% Document structure
\begin{document}

\TITLE{Nicolas Forstner}

\center{
    \begin{tabular}{cc}
        \begin{tabular}{rll}
            \textsc{Mobile:} & +44 7379 400253 \\
            \textsc{Email:}  & \href{mailto:nls.forstner@gmail.com}{nls.forstner@gmail.com} \\
        \end{tabular} &
        \begin{tabular}{rll}
            \textsc{Homepage:} & \href{https:nforstner.com}{nforstner.com} \\
            \textsc{Based in:} & London \\
        \end{tabular}
    \end{tabular}
}

\section{Education}

\begin{tabular}{r|l}
    % UNIVERSITY
    \textsc{2018-2022} &
        \begin{tabularx}{0.8\textwidth}{X}
            \textsc{University of York (Russel Group)} \\

            \emph{MEng (Hons) Computer Science with Artificial
            Intelligence} \\

            \tabitem Recipient of the IET Prize for the most outstanding student
            on an IET accredited course. \\

            \tabitem First class honors in all four years of the
            degree. A complete academic transcript is available on
            \href{https://nforstner.com/transcript.pdf}
            {nforstner.com/transcript.pdf}.

            \tabitem Third year thesis on object tracking using
            both conventional video and optical flow. \\

        \end{tabularx} \\
    \multicolumn{2}{c}{} \\

    % SCHOOL
                  & \textsc{School} \\
    \textsc{2017} &
        \emph{German (Bavarian) \emph{Abitur} with a grade average of 1.6 (1 is
        best, 6 is worst)} \\

\end{tabular}

\section{Work Experience}

\begin{tabular}{r|l}

% UoY Project

    \begin{tabular}{r}
        \textsc{June \&} \\
        \textsc{July} \\
        \textsc{2022} \\
    \end{tabular} &

    \begin{tabularx}{0.8\textwidth}{X}

        \textsc{University of York} \\

        \emph{Lorem Ipsum et dolor set amed.} \\

        \tabitem Lorem Ipsum et dolor set amed. \\

    \end{tabularx} \\

    \multicolumn{2}{c}{} \\

% ATOS 2

    \begin{tabular}{r}
        \textsc{July \&} \\
        \textsc{August} \\
        \textsc{2020} \\
    \end{tabular} &

    \begin{tabularx}{0.8\textwidth}{X}

        \textsc{Atos} \\

        \emph{Internship - Consulting for data science and artificial
        intelligence} \\

        \tabitem Detection of humans through walls based on radio signals as
        part of a project for the German military (i.e. the \emph{BWI}). \\

        \tabitem I was tasked with developing a cross-modal teacher-student
        model and a complex data pipeline involving both traditional
        statistical methods and modern machine learning techniques. \\

        \tabitem As part of our SCRUM cycle, I was regularly chosen to present
        our team's work directly to the customer and to manage further
        communication. \\

    \end{tabularx} \\

    \multicolumn{2}{c}{} \\

% ATOS 1

    \begin{tabular}{r}
        \textsc{September} \\
        \textsc{2019}
    \end{tabular} &

    \begin{tabularx}{0.8\textwidth}{X}
        \textsc{Atos} \\

        \emph{Internship - Consulting for data science and artificial
        intelligence} \\

        \tabitem Development of software that uses machine learning to improve
        the efficiency of a production line for electrical parts in a Siemens
        factory. \\

        \tabitem By making heavy use of data preprocessing and hyper-parameter
        optimization, my approach was able to beat all alternatives and was
        chosen to be deployed into production. \\

    \end{tabularx} \\

    \multicolumn{2}{c}{} \\

% PADBERG

    \begin{tabular}{r}
      \textsc{August \&} \\
      \textsc{September} \\
      \textsc{2018}
    \end{tabular} &

    \begin{tabularx}{0.8\textwidth}{X}
        \textsc{Padberg \& Partners} \\

        \emph{Internship - Web development and task automation} \\

        \tabitem Contributions to the open-source marketing platform
        \textit{Mautics} and development of scripts to automate migrations from
        proprietary software to open-source alternatives. \\

    \end{tabularx}

\end{tabular}

\section{Skills}

\begin{tabular}{ r p{0.73\textwidth} }
        Data Science: & Data visualisation, unstructured data, data mining, data pipelines, exploratory data analysis, presentation and communication \\
        Machine learning: & Artificial neural networks, network architectures, hyper-parameter optimisation, XGBoost, linear regression, PCA, clustering, feature engineering \\
        Python: & Numpy, Pandas, Scikit-Learn, PyTorch, Tensorflow, Keras, Pytest, Jupyter, SQLAlchemy, OpenCV, Flask, Django \\
        JavaScript: & Node, Webpack, Vue.js, Vuex, Vue Router, Vuetify, Axios, Express \\
        Other languages: & Rust, Java, SQL, C, Bash, Haskell, Scheme, Prolog, \LaTeX \\
        Software \& DevOps: & Linux, Git, Docker, Docker-Compose, Nginx, PostgreSQL, SSH, Make, Vim \\
        Cloud: & Amazon Web Services (AWS), Google Cloud (GCP), Digital Ocean \\
\end{tabular}

\section{Languages}
\begin{tabular}{rl}
        \textsc{English:} & Business level \\
        \textsc{German:} & Native \\
\end{tabular}

\section{Other interests}
    Generative adversarial networks, Dev-Ops, continuous integration, game theory, evolutionary algorithms, system design, 3D animation, ultra-marathons, powerlifting, rowing and free diving.
\end{document}
